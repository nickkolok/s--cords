\documentclass[a4paper,12pt]{article} %размер бумаги устанавливаем А4, шрифт 12пунктов
\usepackage[T2A]{fontenc}
\usepackage[utf8]{inputenc}
\usepackage[english,russian]{babel} %используем русский и английский языки с переносами
\usepackage{amssymb,amsfonts,amsmath,mathtext,enumerate,float,amsthm} %подключаем нужные пакеты расширений
\usepackage[unicode,colorlinks=true,citecolor=black,linkcolor=black]{hyperref}
%\usepackage[pdftex,unicode,colorlinks=true,linkcolor=blue]{hyperref}
\usepackage{indentfirst} % включить отступ у первого абзаца
\usepackage[dvips]{graphicx} %хотим вставлять рисунки?
\graphicspath{{illustr/}}%путь к рисункам

\makeatletter
\renewcommand{\@biblabel}[1]{#1.} % Заменяем библиографию с квадратных скобок на точку:
\makeatother %Смысл этих трёх строчек мне непонятен, но поверим "Запискам дебианщика"

\usepackage{geometry} % Меняем поля страницы.
\geometry{left=2cm}% левое поле
\geometry{right=1cm}% правое поле
\geometry{top=2cm}% верхнее поле
\geometry{bottom=2cm}% нижнее поле

\renewcommand{\theenumi}{\arabic{enumi}}% Меняем везде перечисления на цифра.цифра
\renewcommand{\labelenumi}{\arabic{enumi}}% Меняем везде перечисления на цифра.цифра
\renewcommand{\theenumii}{.\arabic{enumii}}% Меняем везде перечисления на цифра.цифра
\renewcommand{\labelenumii}{\arabic{enumi}.\arabic{enumii}.}% Меняем везде перечисления на цифра.цифра
\renewcommand{\theenumiii}{.\arabic{enumiii}}% Меняем везде перечисления на цифра.цифра
\renewcommand{\labelenumiii}{\arabic{enumi}.\arabic{enumii}.\arabic{enumiii}.}% Меняем везде перечисления на цифра.цифра

\sloppy


\begin{document}

Выполнил Николай Авдеев.

\paragraph{Постановка задачи.}

Выписать функцию Грина для краевой задачи, состоящей из уравнения
\begin{equation}
	-(p(x)u'(x))' = f(x)
\end{equation}
с краевыми условиями
\begin{gather}
	l_1(u) = \alpha_{11} u'(0) + \alpha_{10} u(0) = 0
	\\
	l_2(u) = \alpha_{21} u'(l) + \alpha_{20} u(l) = 0
\end{gather}
при $p(x)$ общего вида и при $p(x)\equiv 1$.

\paragraph{Решение.}

\paragraph{Однородное уравнение.}

Выпишем и решим однородное уравенение:
\begin{equation}
	-(p(x)u'(x))' = 0
\end{equation}
или, что то же самое,
\begin{equation}
	p(x)u'(x) = C_1,
\end{equation}
откуда
\begin{equation}
	u(x) = C_1\int_{0}^{x} \frac{d\tau}{p(\tau)} + C_2.
\end{equation}
Следовательно, в качестве фундаментальной системы решений (ФСР) можно взять
\begin{equation}\label{fsr-int}
	u_1(x) \equiv 1, ~~ u_2(x) = \int_{0}^{x} \frac{d\tau}{p(\tau)}.
\end{equation}

\paragraph{Функция Коши.}
Построим сначала функцию $k(x, s)$:
\begin{multline}
	k(x,s) =
	- \frac{
		\left|\begin{array}{cc}
			u_1(s)  & u_2(s) \\
			u_1(x)  & u_2(x)
		\end{array}\right|
	}{
		p(s)
		\left|\begin{array}{cc}
			u_1 (s) & u_2 (s) \\
			u_1'(s) & u_2'(s)
		\end{array}\right|
	}
	=
	- \frac{
		\left|\begin{array}{cc}
			1  & u_2(s) \\
			1  & u_2(x)
		\end{array}\right|
	}{
		p(s)
		\left|\begin{array}{cc}
			1 & u_2 (s) \\
			0 & u_2'(s)
		\end{array}\right|
	}
	=
	- \frac{
		u_2(x) - u_2(s)
	}{
		p(s) u_2'(s)
	}
	=
	- \frac{
		u_2(x) - u_2(s)
	}{
		p(s) \frac{1}{p(s)}
	}
	=
	\\=
	u_2(s) - u_2(x)
	=
	\int_x^s\frac{d\tau}{p(\tau)}
\end{multline}
Заметим, что функция $k(x,s)$ является аддитивной функцией промежутка.

Таким образом, ФСР \eqref{fsr-int} можно записать в виде
\begin{equation}\label{fsr-k}
	u_1(x) \equiv 1, ~~ u_2(x) = k(0,x).
\end{equation}

Теперь мы можем выписать функцию Коши.
Зная, что
\begin{equation}
	K_2(x, s) = \begin{cases}
		0     , & \mbox{если~~~} 0 \leq x \leq s \leq l
		\\
		k(x,s), & \mbox{если~~~} 0 \leq s \leq x \leq l
	\end{cases}
\end{equation}
получаем функцию Коши в явном виде:
\begin{equation}
	K_2(x, s) = \begin{cases}
		0     , & \mbox{если~~~} 0 \leq x \leq s \leq l
		\\
		\int_x^s\frac{d\tau}{p(\tau)}, & \mbox{если~~~} 0 \leq s \leq x \leq l
	\end{cases}
\end{equation}

\paragraph{Характеристический определитель.}

Выпишем характеристический определитель:
\begin{multline}
	\Delta
	=
	\left|\begin{array}{cc}
		l_1 u_1 & l_1 u_2 \\
		l_2 u_1 & l_2 u_2
	\end{array}\right|
	=
	\left|\begin{array}{cc}
		\alpha_{10} & \frac{\alpha_{11}}{p(0)} \\
		\alpha_{20} & \frac{\alpha_{21}}{p(l)} +\alpha_{20} k(0,l)
	\end{array}\right|
	=
	\frac{\alpha_{10}\alpha_{21}}{p(l)} + \alpha_{10}\alpha_{20} k(0,l) - \frac{\alpha_{11}\alpha_{20}}{p(0)}
	=
	\\=
	\frac{\alpha_{10}\alpha_{21}p(0)  - \alpha_{11}\alpha_{20}p(l) + \alpha_{10}\alpha_{20} p(0) p(l) k(0,l)}{p(0)p(l)}
\end{multline}

\paragraph{Специальная ФСР (СФСР)}
Построим СФСР $\{v_1, v_2\}$:
\begin{multline}
	v_1 =
	\Delta^{-1}
	\left|\begin{array}{cc}
		u_1(x) & u_2(x) \\
		l_2 u_1 & l_2 u_2
	\end{array}\right|
	=
	\Delta^{-1}
	\left|\begin{array}{cc}
		1 & k(0,x) \\
		\alpha_{20} & \frac{\alpha_{21}}{p(l)} +\alpha_{20} k(0,l)
	\end{array}\right|
	=
	\\=
	\Delta^{-1}
	\left(
		\frac{\alpha_{21}}{p(l)} +\alpha_{20} k(0,l) - \alpha_{20} k(0,x)
	\right)
	=
	\Delta^{-1}
	\left(
		\frac{\alpha_{21}}{p(l)} +\alpha_{20} k(x,l)
	\right)
\end{multline}

\begin{equation}
	v_2 =
	\Delta^{-1}
	\left|\begin{array}{cc}
		l_1 u_1 & l_1 u_2 \\
		u_1(x) & u_2(x)
	\end{array}\right|
	=
	\Delta^{-1}
	\left|\begin{array}{cc}
		\alpha_{10} & \frac{\alpha_{11}}{p(0)} \\
		1       & k(0, x)
	\end{array}\right|
	=
	\Delta^{-1}
	\left(
		\alpha_{10} k(0, x) - \frac{\alpha_{11}}{p(0)}
	\right)
\end{equation}

\paragraph{Функция Грина.}
Вычислим сначала вспомогательные выражения:

\begin{equation}
	l_1K_2(\cdot, s) =
	\alpha_{11} \frac{d}{dx}K_2(x, s)|_{x=0} + \alpha_{10} K_2(0,s)
	=
	0
%	\alpha_{11} \frac{d}{dx}k(x, s)|_{x=0}
%	=
%	\\=
%	\alpha_{11} \left.\left(\frac{d}{dx} \int_x^s \frac{d\tau}{p(\tau)}\right)\right|_{x=0}
%	=
%	- \frac{\alpha_{11}}{p(0)}
\end{equation}

\begin{multline}
	l_2K_2(\cdot, s) =
	\alpha_{21} \frac{d}{dx}K_2(x, s)|_{x=l} + \alpha_{20} K_2(l,s)
	=
	\alpha_{21} \frac{d}{dx}k(x, s)|_{x=l} + \alpha_{20} k(l,s)
	=
	-\frac{\alpha_{21}}{p(l)} + \alpha_{20} k(l,s)
\end{multline}

Функция Грина задаётся равенством
\begin{equation}
	G(x,s) = K_2(x,s) - l_1 K_2(\cdot, s) \cdot v_1(x) - l_2 K_2(\cdot, s) \cdot v_2(x)
\end{equation}
Пусть сначала $0 \leq x\leq s \leq l$, тогда $K_2(x,s) = 0$ и
\begin{multline}
	G(x,s)
	=
	- l_1 K_2(\cdot, s) \cdot v_1(x) - l_2 K_2(\cdot, s) \cdot v_2(x)
	=
	- 0 - l_2 K_2(\cdot, s) \cdot v_2(x)
	=
	\\=
	- \left( -\frac{\alpha_{21}}{p(l)} + \alpha_{20} k(l,s) \right) \cdot v_2(x)
	=
	\left( \frac{\alpha_{21}}{p(l)} - \alpha_{20} k(l,s) \right) \cdot v_2(x)
	=
	\\=
	\left( \frac{\alpha_{21}}{p(l)} - \alpha_{20} k(l,s) \right) \cdot
	\Delta^{-1}
	\left(
		\alpha_{10} k(0, x) - \frac{\alpha_{11}}{p(0)}
	\right)
	=
	\\=
	\Delta^{-1}
	\left(
		\frac{\alpha_{21}}{p(l)} - \alpha_{20} k(l,s)
	\right) \cdot \left(
		\alpha_{10} k(0, x) - \frac{\alpha_{11}}{p(0)}
	\right)
	=
	\\=
	\Delta^{-1}
	\left(
		 \frac{\alpha_{10}\alpha_{21} k(0, x)}{p(l)}
		-\frac{\alpha_{11}\alpha_{21}}{p(0)p(l)}
		-\alpha_{10} \alpha_{20} k(0, x) k(l,s)
		+\frac{\alpha_{11}\alpha_{20} k(l,s)}{p(0)}
	\right)
	=
	\\=
	\frac{
		  \alpha_{10} \alpha_{21} p(0) k(0, x)
		- \alpha_{11} \alpha_{21}
		- \alpha_{10} \alpha_{20} p(0) p(l) k(0, x) k(l, s)
		+ \alpha_{11} \alpha_{20} p(l) k(l,s)
	}{
		\alpha_{10}\alpha_{21}p(0)  - \alpha_{11}\alpha_{20}p(l) + \alpha_{10}\alpha_{20} p(0) p(l) k(0,l)
	}
	=
	\\=
	\frac{
		  \alpha_{10} \alpha_{21} p(0) k(0, x)
		- \alpha_{11} \alpha_{21}
		+ \alpha_{10} \alpha_{20} p(0) p(l) k(0, x) k(s, l)
		- \alpha_{11} \alpha_{20} p(l) k(s,l)
	}{
		\alpha_{10}\alpha_{21}p(0)  - \alpha_{11}\alpha_{20}p(l) + \alpha_{10}\alpha_{20} p(0) p(l) k(0,l)
	}
\end{multline}
Итак, если $0 \leq x\leq s \leq l$, то
\begin{equation}\label{Gxs_x_leq_s}
	G(x,s)
	=
	\frac{
		  \alpha_{10} \alpha_{21} p(0) k(0, x)
		- \alpha_{11} \alpha_{21}
		+ \alpha_{10} \alpha_{20} p(0) p(l) k(0, x) k(s, l)
		- \alpha_{11} \alpha_{20} p(l) k(s,l)
	}{
		\alpha_{10}\alpha_{21}p(0)  - \alpha_{11}\alpha_{20}p(l) + \alpha_{10}\alpha_{20} p(0) p(l) k(0,l)
	}
\end{equation}

Пусть теперь $0 \leq s\leq x \leq l$.

Проведём некоторые подготовительные вычисления:
\begin{multline}
	k(0, x)k(s, l) - k(0,l) k(x,s)
	=
	k(0, x)k(s, l) - k(0,x) k(x,s) - k(x,l) k(x,s)
	=
	\\=
	k(0, x)k(x, l) - k(x,l) k(x,s)
	=
	k(0, s)k(x, l)
\end{multline}

Теперь можно выписать $G(x,s)$ при $0 \leq s\leq x \leq l$,
учитывая, что в таком случае $K_2(x,s) = k(x,s)$:
\begin{multline}
	G(x,s)
	=
	k(x,s)
	+
	\frac{
		  \alpha_{10} \alpha_{21} p(0) k(0, x)
		- \alpha_{11} \alpha_{21}
		+ \alpha_{10} \alpha_{20} p(0) p(l) k(0, x) k(s, l)
		- \alpha_{11} \alpha_{20} p(l) k(s,l)
	}{
		\alpha_{10}\alpha_{21}p(0)  - \alpha_{11}\alpha_{20}p(l) + \alpha_{10}\alpha_{20} p(0) p(l) k(0,l)
	}
	=
	\\=
	\Bigl(
		  \alpha_{10} \alpha_{21} p(0) k(0, x)
		- \alpha_{11} \alpha_{21}
		+ \alpha_{10} \alpha_{20} p(0) p(l) k(0, x) k(s, l)
		- \alpha_{11} \alpha_{20} p(l) k(s, l)
		+ \\
		+ \alpha_{10} \alpha_{21} p(0) k(x, s)
		- \alpha_{11} \alpha_{20} p(l) k(x, s)
		+ \alpha_{10} \alpha_{20} p(0) p(l) k(0,l) k(x,s)
	\Bigr)\cdot \\ \cdot \Bigl(
		\alpha_{10}\alpha_{21}p(0)  - \alpha_{11}\alpha_{20}p(l) + \alpha_{10}\alpha_{20} p(0) p(l) k(0,l)
	\Bigr)^{-1}
\end{multline}


\end{document}
