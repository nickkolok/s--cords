\documentclass[a4paper,12pt]{article} %размер бумаги устанавливаем А4, шрифт 12пунктов
\usepackage[T2A]{fontenc}
\usepackage[utf8]{inputenc}
\usepackage[english,russian]{babel} %используем русский и английский языки с переносами
\usepackage{amssymb,amsfonts,amsmath,mathtext,enumerate,float,amsthm} %подключаем нужные пакеты расширений
\usepackage[unicode,colorlinks=true,citecolor=black,linkcolor=black]{hyperref}
%\usepackage[pdftex,unicode,colorlinks=true,linkcolor=blue]{hyperref}
\usepackage{indentfirst} % включить отступ у первого абзаца
\usepackage[dvips]{graphicx} %хотим вставлять рисунки?
\graphicspath{{illustr/}}%путь к рисункам

\makeatletter
\renewcommand{\@biblabel}[1]{#1.} % Заменяем библиографию с квадратных скобок на точку:
\makeatother %Смысл этих трёх строчек мне непонятен, но поверим "Запискам дебианщика"

\usepackage{geometry} % Меняем поля страницы.
\geometry{left=2cm}% левое поле
\geometry{right=1cm}% правое поле
\geometry{top=2cm}% верхнее поле
\geometry{bottom=2cm}% нижнее поле

\renewcommand{\theenumi}{\arabic{enumi}}% Меняем везде перечисления на цифра.цифра
\renewcommand{\labelenumi}{\arabic{enumi}}% Меняем везде перечисления на цифра.цифра
\renewcommand{\theenumii}{.\arabic{enumii}}% Меняем везде перечисления на цифра.цифра
\renewcommand{\labelenumii}{\arabic{enumi}.\arabic{enumii}.}% Меняем везде перечисления на цифра.цифра
\renewcommand{\theenumiii}{.\arabic{enumiii}}% Меняем везде перечисления на цифра.цифра
\renewcommand{\labelenumiii}{\arabic{enumi}.\arabic{enumii}.\arabic{enumiii}.}% Меняем везде перечисления на цифра.цифра

\sloppy


\begin{document}

Выполнил Николай Авдеев.

\paragraph{Постановка задачи.}

Выписать функцию Грина для краевой задачи, состоящей из уравнения
\begin{equation}
	-(p(x)u'(x))' = f(x)
\end{equation}
с краевыми условиями
\begin{gather}
	l_1(u) = \alpha_{11} u'(0) + \alpha_{10} u(0) = 0
	\\
	l_2(u) = \alpha_{21} u'(l) + \alpha_{20} u(l) = 0
\end{gather}
при $p(x)$ общего вида и при $p(x)\equiv 1$.

\paragraph{Решение.}

\paragraph{Однородное уравнение.}

Выпишем и решим однородное уравенение:
\begin{equation}
	-(p(x)u'(x))' = 0
\end{equation}
или, что то же самое,
\begin{equation}
	p(x)u'(x) = C_1,
\end{equation}
откуда
\begin{equation}
	u(x) = C_1\int_{0}^{x} \frac{d\tau}{p(\tau)} + C_2.
\end{equation}
Следовательно, в качестве фундаментальной системы решений (ФСР) можно взять
\begin{equation}\label{fsr-int}
	u_1(x) \equiv 1, ~~ u_2(x) = \int_{0}^{x} \frac{d\tau}{p(\tau)}.
\end{equation}

\paragraph{Функция Коши.}
Построим сначала функцию $k(x, s)$:
\begin{multline}
	k(x,s) =
	- \frac{
		\left|\begin{array}{cc}
			u_1(s)  & u_2(s) \\
			u_1(x)  & u_2(x)
		\end{array}\right|
	}{
		p(s)
		\left|\begin{array}{cc}
			u_1 (s) & u_2 (s) \\
			u_1'(s) & u_2'(s)
		\end{array}\right|
	}
	=
	- \frac{
		\left|\begin{array}{cc}
			1  & u_2(s) \\
			1  & u_2(x)
		\end{array}\right|
	}{
		p(s)
		\left|\begin{array}{cc}
			1 & u_2 (s) \\
			0 & u_2'(s)
		\end{array}\right|
	}
	=
	- \frac{
		u_2(x) - u_2(s)
	}{
		p(s) u_2'(s)
	}
	=
	- \frac{
		u_2(x) - u_2(s)
	}{
		p(s) \frac{1}{p(s)}
	}
	=
	\\=
	u_2(s) - u_2(x)
	=
	\int_x^s\frac{d\tau}{p(\tau)}
\end{multline}
Заметим, что функция $k(x,s)$ является аддитивной функцией промежутка.

Таким образом, ФСР \eqref{fsr-int} можно записать в виде
\begin{equation}\label{fsr-k}
	u_1(x) \equiv 1, ~~ u_2(x) = k(0,x).
\end{equation}

Теперь мы можем выписать функцию Коши.
Зная, что
\begin{equation}
	K_2(x, s) = \begin{cases}
		0     , & \mbox{если~~~} 0 \leq x \leq s \leq l
		\\
		k(x,s), & \mbox{если~~~} 0 \leq s \leq x \leq l
	\end{cases}
\end{equation}
получаем функцию Коши в явном виде:
\begin{equation}
	K_2(x, s) = \begin{cases}
		0     , & \mbox{если~~~} 0 \leq x \leq s \leq l
		\\
		\int_x^s\frac{d\tau}{p(\tau)}, & \mbox{если~~~} 0 \leq s \leq x \leq l
	\end{cases}
\end{equation}


\end{document}
