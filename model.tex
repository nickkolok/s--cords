\documentclass[a4paper,12pt]{article} %размер бумаги устанавливаем А4, шрифт 12пунктов
\usepackage[T2A]{fontenc}
\usepackage[utf8]{inputenc}
\usepackage[english,russian]{babel} %используем русский и английский языки с переносами
\usepackage{amssymb,amsfonts,amsmath,mathtext,enumerate,float,amsthm} %подключаем нужные пакеты расширений
\usepackage[unicode,colorlinks=true,citecolor=black,linkcolor=black]{hyperref}
%\usepackage[pdftex,unicode,colorlinks=true,linkcolor=blue]{hyperref}
\usepackage{indentfirst} % включить отступ у первого абзаца
\usepackage[dvips]{graphicx} %хотим вставлять рисунки?
\graphicspath{{illustr/}}%путь к рисункам

\makeatletter
\renewcommand{\@biblabel}[1]{#1.} % Заменяем библиографию с квадратных скобок на точку:
\makeatother %Смысл этих трёх строчек мне непонятен, но поверим "Запискам дебианщика"

\usepackage{geometry} % Меняем поля страницы.
\geometry{left=2cm}% левое поле
\geometry{right=1cm}% правое поле
\geometry{top=2cm}% верхнее поле
\geometry{bottom=2cm}% нижнее поле

\renewcommand{\theenumi}{\arabic{enumi}}% Меняем везде перечисления на цифра.цифра
\renewcommand{\labelenumi}{\arabic{enumi}}% Меняем везде перечисления на цифра.цифра
\renewcommand{\theenumii}{.\arabic{enumii}}% Меняем везде перечисления на цифра.цифра
\renewcommand{\labelenumii}{\arabic{enumi}.\arabic{enumii}.}% Меняем везде перечисления на цифра.цифра
\renewcommand{\theenumiii}{.\arabic{enumiii}}% Меняем везде перечисления на цифра.цифра
\renewcommand{\labelenumiii}{\arabic{enumi}.\arabic{enumii}.\arabic{enumiii}.}% Меняем везде перечисления на цифра.цифра

\sloppy


\begin{document}

\setlength{\jot}{12pt}

Выполнил Николай Авдеев. Вариант 2.

\begin{figure}[ht]
	\includegraphics[width=\textwidth]{scheme.pdf}
	%\caption{}
\end{figure}

\paragraph{Обозначения.}
Левый конец ребра $\gamma_i$ будем обозначать через $\gamma_i^-$,
правый~--- через $\gamma_i^+$.
Значения функции в этих точках будем понимать в смысле соответствующего одностороннего предела.

\paragraph{Функционал равновесия.}
Выписываем сначала функционал энергии.
Энергия, накопленная в струнах, учитывается так же, как и в случае одной струны.
Энергия, накопленная в пружинах, учитывается по закону Гука.
\begin{multline}
	Vu = \sum_{i=1}^4 V_{i_{\mbox{\tiny струны}}}u + \sum_{i=1}^4 V_{i_{\mbox{\tiny пружины}}}u
	=
	\\=
	\int_{\Gamma} \frac{p(x)(u'(x))^2}{2} dx
	+ \frac{k_1}{2} (u(a_1^1))^2
	+ \frac{k_2}{2} (u(b_2))^2
	+ \frac{k_3}{2} (u(a_2^3)-u(a_2^4))^2
	+ \frac{k_4}{2} (u(a_2^2)-u(a_2^4))^2
\end{multline}
Функционал работы выписывается аналогично одномерному случаю:
\begin{equation}
	A_f u = \int_{\Gamma} f(x) u(x) dx
\end{equation}
Следовательно, функционал равновесия имеет вид:
\begin{multline}
	Ju =
	Vu - A_f u
	=
	\\=
	\int_{\Gamma} \frac{p(x)(u'(x))^2}{2} - f(x) u(x) dx
	+ \frac{k_1}{2} (u(a_1^1))^2
	+ \frac{k_2}{2} (u(b_2))^2
	+ \frac{k_3}{2} (u(a_2^3)-u(a_2^4))^2
	+ \frac{k_4}{2} (u(a_2^2)-u(a_2^4))^2
\end{multline}
Найдём теперь его первую вариацию.
В силу линейности первой вариации мы можем воспользоваться
выражением для первой вариации интегральной части, полученным ранее:
\begin{multline}
	\delta\left( \int_{\Gamma} \frac{p(x)(u'(x))^2}{2} - f(x) u(x) dx, v \right)
	=
	\int_{\Gamma} p(x)u'(x)v'(x) - f(x) v(x) dx
	=
	\\=
	\left.\sum_{i=1}^{4} p(x) u'(x) v(x)\right|_{x=\gamma_i^-}^{x=\gamma_i^+}
	=
	\\ =
	  p(a_1^1) u'(a_1^1) v(a_1^1) - p(b_1  ) u'(b_1  ) v(b_1  )
	+ p(b_2  ) u'(b_2  ) v(b_2  ) - p(a_2^2) u'(a_2^2) v(a_2^2)
	+ \\
	+ p(b_3  ) u'(b_3  ) v(b_3  ) - p(a_2^3) u'(a_2^3) v(a_2^3)
	+ p(a_2^4) u'(a_2^4) v(a_2^4) - p(a_1^4) u'(a_1^4) v(a_1^4)
\end{multline}
Исходя из того, что
\begin{equation}
	\delta((u(a))^2, v) = 2 u(a) v(a)
\end{equation}
и
\begin{equation}
	\delta((u(a)-u(b))^2, v) = 2 (u(a) - u(b)) (v(a) - v(b))
\end{equation}
можно выписать, что
\begin{multline}
	\delta(Ju, v)
	=
	  p(a_1^1) u'(a_1^1) v(a_1^1) - p(b_1  ) u'(b_1  ) v(b_1  )
	+ p(b_2  ) u'(b_2  ) v(b_2  ) - p(a_2^2) u'(a_2^2) v(a_2^2)
	+ \\
	+ p(b_3  ) u'(b_3  ) v(b_3  ) - p(a_2^3) u'(a_2^3) v(a_2^3)
	+ p(a_2^4) u'(a_2^4) v(a_2^4) - p(a_1^4) u'(a_1^4) v(a_1^4)
	+ \\
	+ k_1  u(a_1^1) v(a_1^1)
	+ k_2  u(b_2  ) v(b_2  )
	+ k_3 (u(a_2^3) - u(a_2^4)) (v(a_2^3) - v(a_2^4))
	+ k_4 (u(a_2^2) - u(a_2^4)) (v(a_2^2) - v(a_2^4))
\end{multline}

Из условия жёсткого закрепления $\gamma_1$ в $b_1$ имеем
\begin{equation}
	u(b_1) = v(b_1) = 0.
\end{equation}

Из условия свободного конца $\gamma_3$ в $b_3$ имеем
\begin{equation}
	u'(b_3) = v'(b_3) = 0.
\end{equation}

Из условия сцепления струн $\gamma_1$ и $\gamma_4$ в точке $a_1$ имеем
\begin{equation}
	u(a_1^1) = u(a_1^4), ~~ v(a_1^1) = v(a_1^4)
\end{equation}

С учётом этих трёх условий имеем
\begin{multline}
	\delta(Ju, v)
	=
	  p(a_1^1) u'(a_1^1) v(a_1^1)
	+ p(b_2  ) u'(b_2  ) v(b_2  ) - p(a_2^2) u'(a_2^2) v(a_2^2)
	- \\
	                              - p(a_2^3) u'(a_2^3) v(a_2^3)
	+ p(a_2^4) u'(a_2^4) v(a_2^4) - p(a_1^4) u'(a_1^4) v(a_1^1)
	+ \\
	+ k_1  u(a_1^1) v(a_1^1)
	+ k_2  u(b_2  ) v(b_2  )
	+ k_3 (u(a_2^3) - u(a_2^4)) (v(a_2^3) - v(a_2^4))
	+ k_4 (u(a_2^2) - u(a_2^4)) (v(a_2^2) - v(a_2^4))
\end{multline}



Воспользуемся теперь произвольностью выбора $v$
и тем, что в стационарном состоянии $\delta(Ju, v) = 0$.
Обозначим $A = \{b_2, b_3, a_1^1, a_1^4, a_2^2, a_2^3, a_2^4\}$.
Возьмём сначала $v$ такую, что $v(A \setminus \{a_1^1\}) = 0$, $v(a_1^1) \neq 0$.
Тогда
\begin{multline*}
	0 = \delta(Ju, v) =
	  p(a_1^1) u'(a_1^1) v(a_1^1)
	- p(a_1^4) u'(a_1^1) v(a_1^1)
	+ k_1  u(a_1^1) v(a_1^1)
	= \\ =
	  p(a_1^1) u'(a_1^1) v(a_1^1)
	- p(a_1^4) u'(a_1^4) v(a_1^1)
	+ k_1  u(a_1^1) v(a_1^1)
	,
\end{multline*}
откуда
\begin{equation}
	  p(a_1^1) u'(a_1^1)
	- p(a_1^4) u'(a_1^4)
	+ k_1  u(a_1^1) = 0
\end{equation}

Возьмём $v$ такую, что $v(A \setminus \{a_2^2\}) = 0$, $v(a_2^2) \neq 0$.
Тогда
\begin{equation*}
	0 = \delta(Ju, v) =
	- p(a_2^2) u'(a_2^2) v(a_2^2)
	+ k_4 (u(a_2^2) - u(a_2^4)) v(a_2^2)
	,
\end{equation*}
откуда
\begin{equation}\label{l_6_pre}
	- p(a_2^2) u'(a_2^2) + k_4 (u(a_2^2) - u(a_2^4)) = 0
\end{equation}


Возьмём $v$ такую, что $v(A \setminus \{a_2^3\}) = 0$, $v(a_2^3) \neq 0$.
Тогда
\begin{equation*}
	0 = \delta(Ju, v) =
	                              - p(a_2^3) u'(a_2^3) v(a_2^3)
	+ k_3 (u(a_2^3) - u(a_2^4)) v(a_2^3)
	,
\end{equation*}
откуда
\begin{equation}\label{l_7_pre}
	- p(a_2^3) u'(a_2^3) + k_3 (u(a_2^3) - u(a_2^4)) = 0
\end{equation}



Возьмём $v$ такую, что $v(A \setminus \{a_2^4\}) = 0$, $v(a_2^4) \neq 0$.
Тогда
\begin{equation*}
	0 = \delta(Ju, v) =
	  p(a_2^4) u'(a_2^4) v(a_2^4)
	- k_3 (u(a_2^3) - u(a_2^4))  v(a_2^4)
	- k_4 (u(a_2^2) - u(a_2^4))  v(a_2^4)
	,
\end{equation*}
откуда
\begin{equation}
	p(a_2^4) u'(a_2^4) - k_3 (u(a_2^3) - u(a_2^4)) - k_4 (u(a_2^2) - u(a_2^4)) = 0
\end{equation}
и с учётом \eqref{l_6_pre} и \eqref{l_7_pre} получаем
\begin{equation}
	p(a_2^4) u'(a_2^4) - p(a_2^3) u'(a_2^3) - p(a_2^2) u'(a_2^2) = 0
\end{equation}


Наконец, возьмём $v$ такую, что $v(A \setminus \{b_2\}) = 0$, $v(b_2) \neq 0$.
Тогда
\begin{equation*}
	0 = \delta(Ju, v) =
	  p(b_2  ) u'(b_2  ) v(b_2  )
	+ k_2  u(b_2  ) v(b_2  )
	,
\end{equation*}
откуда
\begin{equation}
	p(b_2  ) u'(b_2  ) + k_2  u(b_2  ) = 0
\end{equation}

\paragraph{Математическая модель.}
Таким образом, математической моделью исследуемой струнно-пружинной системы является
краевая задача, состоящая из уравнения
\begin{equation}
	-(p(x)u'(x))' = f(x)
\end{equation}
и следующих 8 условий:
\begin{align}
	l_1 u & = u(b_1) = 0
	\\
	l_2 u & = p(b_2  ) u'(b_2  ) + k_2  u(b_2  ) = 0
	\\
	l_3 u & = u'(b_3) = 0
	\\
	l_4 u & = u(a_1^1) - u(a_1^4) = 0
	\\
	l_5 u & = p(a_1^1) u'(a_1^1) - p(a_1^4) u'(a_1^4) + k_1  u(a_1^1) = 0
	\\
	l_6 u & = - p(a_2^2) u'(a_2^2) + k_4 (u(a_2^2) - u(a_2^4)) = 0
	\\
	l_7 u & = - p(a_2^3) u'(a_2^3) + k_3 (u(a_2^3) - u(a_2^4)) = 0
	\\
	l_8 u & = p(a_2^4) u'(a_2^4) - p(a_2^3) u'(a_2^3) - p(a_2^2) u'(a_2^2) = 0
\end{align}

\paragraph{Фундаментальная система решений (ФСР).}
Пусть
\begin{equation}
	\chi_{i}(x) = \begin{cases}
		1, & \mbox{~если~} x \in \gamma_i
		\\
		0, & \mbox{~если~} x \notin \gamma_i.
	\end{cases}
\end{equation}
и пусть
\begin{equation}
	P(x,y) = \int_x^y \frac{d\tau}{p(\tau)}
\end{equation}
(в силу того, что рассматриваемый граф является деревом, такая запись не приводит к недоразумению).
Тогда следующая система функций:
\begin{align}
	u_{i1}(x) & = \chi_i(x)
	\\
	u_{12}(x) & = \chi_1(x) P(b_1  , x)
	\\
	u_{22}(x) & = \chi_2(x) P(b_2  , x)
	\\
	u_{32}(x) & = \chi_3(x) P(a_2^3, x)
	\\
	u_{42}(x) & = \chi_4(x) P(a_2^4, x)
\end{align}
Такой выбор ФСР обусловлен стремлением обратить в нуль
как можно больше членов характеристического определителя $\Delta$,
к которому мы сейчас и переходим.

\paragraph{Характеристический определитель.}
Построим характеристический определитель
\begin{equation}
	\Delta = | (l_i u_{j1}, l_i u_{j2})|.
\end{equation}

Введём сначала параметризацию.
Будем полагать, что каждое ребро параметризовано от 0 до 1 слева направо.
Масштабирование параметра, обеспечиваюшее произвольную длину ребра,
можно будет провести на конечном этапе
(а при вычислениях и вовсе возложить на ЭВМ).
Далее будем отождествлять $x$ с параметром.

Для каждого $i$ выпишем те функции из ФСР, на которых функционал $l_i$
принимает ненулевое значение.

\begin{align}
	l_1u_{11} & = 1
	\\
	l_2u_{21} & = k_2
	\\
	l_2u_{22} & = \frac{p(b_2)}{-p(b_2)} = -1
	\\
	l_3u_{32} & = \frac{1}{p(b_3)}
	\\
	l_4u_{11} & = 1
	\\
	l_4u_{41} & = -1
	\\
	l_4u_{12} & = P(b_1, a_1^1)
	\\
	l_4u_{42} & = - P(a_2^4, a_1^4)
	\\
	% l_5 u & = p(a_1^1) u'(a_1^1) - p(a_1^4) u'(a_1^4) + k_1  u(a_1^1)
	l_5u_{11} & = k_1
	\\
	l_5u_{12} & = 1 + k_1  P(b_1, a_1^1)
	\\
	l_5u_{42} & = - p(a_1^4) \cdot \left(-\frac{1}{p(a_1^4)}\right) = 1
	\\
	% l_6 u & = - p(a_2^2) u'(a_2^2) + k_4 (u(a_2^2) - u(a_2^4))
	l_6u_{21} & = k_4
	\\
	l_6u_{22} & = - p(a_2^2) \cdot \left(-\frac{1}{p(a_2^2)}\right) + k_4 P(b_2, a_2^2) = 1 + k_4 P(b_2, a_2^2)
	\\
	l_6u_{41} & = - k_4
	\\
	%l_6u_{42} & = 0
	%\\
	% l_7 u & = - p(a_2^3) u'(a_2^3) + k_3 (u(a_2^3) - u(a_2^4))
	l_7u_{31} & = k_3
	\\
	l_7u_{32} & = - 1
	\\
	l_7u_{41} & = - k_3
	\\
	%l_7u_{42} & = 0
	%\\
	% l_8 u & = p(a_2^4) u'(a_2^4) - p(a_2^3) u'(a_2^3) - p(a_2^2) u'(a_2^2)
	l_8u_{22} & = - p(a_2^2) \cdot \frac{-1}{p(a_2^2)} =  1
	\\
	l_8u_{32} & = - p(a_2^3) \cdot \frac{ 1}{p(a_2^3)} = -1
	\\
	l_8u_{42} & =   p(a_2^4) \cdot \frac{-1}{p(a_2^4)} = -1
\end{align}

Теперь с помощью wxMaxima несложно найти характеристический определитель:
\begin{multline}\label{charact_det}
	\Delta =
	\frac{k_3}{p(b_3)}
	%\cdot \\ \cdot
	\left(
		\left( -1-P\left( a_2^4,a_1^4\right)  k_2-P\left( b_2,a_2^2\right)  k_2\right)
		\left( k_4+P\left( b_1,a_1^1\right)  k_1 k_4\right)
		\right. + \\ + \left.
		k_2 \left( -1-P\left( b_1,a_1^1\right)  k_1-P\left( b_1,a_1^1\right)  k_4\right)
	\right)
\end{multline}

\paragraph{Невырожденность задачи.}
Из \eqref{charact_det} легко видеть,
что при $k_3 = 0$ задача становится вырожденной.
Обнуление жёсткости других пружин (т.е. де-факто замена их на кольца)
приводит к не столь явным результатам.

Пусть $k_2 = 0$, тогда
\begin{equation}
	\Delta = - \frac{k_3}{p(b_3)} \left( P\left( b_1,a_1^1\right)  k_1+1\right)  k_4
\end{equation}

Если же $k_4 = 0$, то
\begin{equation}
	\Delta = - \frac{k_3}{p(b_3)} \left( P\left( b_1,a_1^1\right)  k_1+1\right)  k_4
\end{equation}


\paragraph{Специальная фундаментальная система решений (СФСР).}
Найдём СФСР $\{v_i(x)\}_{i=1}^{8}$.
У всех функций $v_i(x)$ будет общий знаменатель,
наличие которого объясняется тем,
что в формуле для СФСР присутсвует деление на характеристичекий определитель $\Delta$.
Однако этот знаменатель (обозначим его $\tilde{\Delta}$)
не совпадает с $\Delta$,
т.к. некоторые множители сокращаются.


Итак,
\begin{multline}
	\tilde{\Delta} =
	\left( k_2 k_4+k_1 k_2 k_4 \int_{2}^{3}\frac{dt}{p(t)}\right)  \int_{14}^{15}\frac{dt}{p(t)}+
	\\+
	\left( k_2 k_4+k_1 k_2 k_4 \int_{2}^{3}\frac{dt}{p(t)}\right)  \int_{6}^{7}\frac{dt}{p(t)}+
	\\+
	\left( k_1 k_2+\left( k_2+k_1\right)  k_4\right)  \int_{2}^{3}\frac{dt}{p(t)}+k_4+k_2
\end{multline}

\begin{multline}
	\frac{v_1(x)}{\tilde{\Delta}}=
	(k_2+k_4+\left( \left( k_1+k_2\right)  k_4+k_1 k_2\right)  \int_{2}^{3}\frac{dt}{p(t)}+\left( \left( -k_1-k_2\right)  k_4-k_1 k_2\right)  \int_{2}^{x}\frac{dt}{p(t)}+\left( -k_1 k_2 k_4 \int_{2}^{x}\frac{dt}{p(t)}+k_1 k_2 k_4 \int_{2}^{3}\frac{dt}{p(t)}+k_2 k_4\right)  \int_{6}^{7}\frac{dt}{p(t)}+\left( -k_1 k_2 k_4 \int_{2}^{x}\frac{dt}{p(t)}+k_1 k_2 k_4 \int_{2}^{3}\frac{dt}{p(t)}+k_2 k_4\right)  \int_{14}^{15}\frac{dt}{p(t)}
\end{multline}

\begin{multline}
	\frac{v_2(x)}{\tilde{\Delta}}=
	k_4 \int_{2}^{x}\frac{dt}{p(t)} thechi_4^1\left( x\right) +\left( 1+\left( k_4+k_1\right)  \int_{2}^{3}\frac{dt}{p(t)}+\left( k_1 k_4 \int_{2}^{3}\frac{dt}{p(t)}+k_4\right)  \int_{6}^{7}\frac{dt}{p(t)}+\left( k_1 k_4 \int_{2}^{3}\frac{dt}{p(t)}+k_4\right)  \int_{14}^{x}\frac{dt}{p(t)}\right)  thechi_3^2\left( x\right) +\left( k_4 \int_{2}^{3}\frac{dt}{p(t)}+\left( k_1 k_4 \int_{2}^{3}\frac{dt}{p(t)}+k_4\right)  \int_{6}^{7}\frac{dt}{p(t)}\right)  thechi_2^3\left( x\right) +\left( k_4 \int_{2}^{3}\frac{dt}{p(t)}+\left( k_1 k_4 \int_{2}^{3}\frac{dt}{p(t)}+k_4\right)  \int_{6}^{x}\frac{dt}{p(t)}\right)  thechi_1^4\left( x\right)
\end{multline}

\begin{multline}
	\frac{v_3(x)}{\tilde{\Delta}}=
	\left( \left( p\left( 11\right)  k_4+p\left( 11\right)  k_2\right)  \int_{2}^{x}\frac{dt}{p(t)}+p\left( 11\right)  k_2 k_4 \int_{2}^{x}\frac{dt}{p(t)} \int_{14}^{15}\frac{dt}{p(t)}\right)  thechi_4^1\left( x\right) + thechi_3^2\left( x\right) ++ thechi_1^4\left( x\right)
\end{multline}

\begin{multline}
	\frac{v_4(x)}{\tilde{\Delta}}=
	k_2 k_4 \int_{2}^{x}\frac{dt}{p(t)} thechi_4^1\left( x\right) +\left( -k_4-k_1 k_4 \int_{2}^{3}\frac{dt}{p(t)}+\left( -k_1 k_2 k_4 \int_{2}^{3}\frac{dt}{p(t)}-k_2 k_4\right)  \int_{14}^{15}\frac{dt}{p(t)}+\left( k_1 k_2 k_4 \int_{2}^{3}\frac{dt}{p(t)}+k_2 k_4\right)  \int_{14}^{x}\frac{dt}{p(t)}\right)  thechi_3^2\left( x\right) +\left( -k_2-k_4+\left( -k_1 k_4-k_1 k_2\right)  \int_{2}^{3}\frac{dt}{p(t)}+\left( -k_1 k_2 k_4 \int_{2}^{3}\frac{dt}{p(t)}-k_2 k_4\right)  \int_{14}^{15}\frac{dt}{p(t)}\right)  thechi_2^3\left( x\right) + thechi_1^4\left( x\right)
\end{multline}

\begin{multline}
	\frac{v_5(x)}{\tilde{\Delta}}=
	\left( \left( k_4+k_2\right)  \int_{2}^{x}\frac{dt}{p(t)}+k_2 k_4 \int_{2}^{x}\frac{dt}{p(t)} \int_{6}^{7}\frac{dt}{p(t)}+k_2 k_4 \int_{2}^{x}\frac{dt}{p(t)} \int_{14}^{15}\frac{dt}{p(t)}\right)  thechi_4^1\left( x\right) +\left( k_4 \int_{2}^{3}\frac{dt}{p(t)}+k_2 k_4 \int_{2}^{3}\frac{dt}{p(t)} \int_{14}^{15}\frac{dt}{p(t)}-k_2 k_4 \int_{2}^{3}\frac{dt}{p(t)} \int_{14}^{x}\frac{dt}{p(t)}\right)  thechi_3^2\left( x\right) +\left( \left( k_4+k_2\right)  \int_{2}^{3}\frac{dt}{p(t)}+k_2 k_4 \int_{2}^{3}\frac{dt}{p(t)} \int_{14}^{15}\frac{dt}{p(t)}\right)  thechi_2^3\left( x\right) +\left( \left( k_4+k_2\right)  \int_{2}^{3}\frac{dt}{p(t)}+k_2 k_4 \int_{2}^{3}\frac{dt}{p(t)} \int_{6}^{7}\frac{dt}{p(t)}-k_2 k_4 \int_{2}^{3}\frac{dt}{p(t)} \int_{6}^{x}\frac{dt}{p(t)}+k_2 k_4 \int_{2}^{3}\frac{dt}{p(t)} \int_{14}^{15}\frac{dt}{p(t)}\right)  thechi_1^4\left( x\right)
\end{multline}

\begin{multline}
	\frac{v_6(x)}{\tilde{\Delta}}=
	-k_2 \int_{2}^{x}\frac{dt}{p(t)} thechi_4^1\left( x\right) +\left( 1+k_1 \int_{2}^{3}\frac{dt}{p(t)}+\left( k_1 k_2 \int_{2}^{3}\frac{dt}{p(t)}+k_2\right)  \int_{14}^{15}\frac{dt}{p(t)}+\left( -k_1 k_2 \int_{2}^{3}\frac{dt}{p(t)}-k_2\right)  \int_{14}^{x}\frac{dt}{p(t)}\right)  thechi_3^2\left( x\right) +\left( \left( -k_1 k_2 \int_{2}^{3}\frac{dt}{p(t)}-k_2\right)  \int_{6}^{7}\frac{dt}{p(t)}-k_2 \int_{2}^{3}\frac{dt}{p(t)}\right)  thechi_2^3\left( x\right) +\left( \left( -k_1 k_2 \int_{2}^{3}\frac{dt}{p(t)}-k_2\right)  \int_{6}^{x}\frac{dt}{p(t)}-k_2 \int_{2}^{3}\frac{dt}{p(t)}\right)  thechi_1^4\left( x\right)
\end{multline}

\begin{multline}
	\frac{v_7(x)*k_3}{\tilde{\Delta}}=
	(\left( k_2 k_4+k_1 k_2 k_4 \int_{2}^{3}\frac{dt}{p(t)}\right)  \int_{14}^{15}\frac{dt}{p(t)}+\left( k_2 k_4+k_1 k_2 k_4 \int_{2}^{3}\frac{dt}{p(t)}\right)  \int_{6}^{7}\frac{dt}{p(t)}+\left( k_1 k_2+\left( k_2+k_1\right)  k_4\right)  \int_{2}^{3}\frac{dt}{p(t)}+k_4+k_2
\end{multline}

\begin{multline}
	\frac{v_8(x)}{\tilde{\Delta}}=
	\left( \left( k_4+k_2\right)  \int_{2}^{x}\frac{dt}{p(t)}+k_2 k_4 \int_{2}^{x}\frac{dt}{p(t)} \int_{14}^{15}\frac{dt}{p(t)}\right)  thechi_4^1\left( x\right) + thechi_3^2\left( x\right) + thechi_2^3\left( x\right) + thechi_1^4\left( x\right)
\end{multline}


\end{document}
