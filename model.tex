\documentclass[a4paper,12pt]{article} %размер бумаги устанавливаем А4, шрифт 12пунктов
\usepackage[T2A]{fontenc}
\usepackage[utf8]{inputenc}
\usepackage[english,russian]{babel} %используем русский и английский языки с переносами
\usepackage{amssymb,amsfonts,amsmath,mathtext,enumerate,float,amsthm} %подключаем нужные пакеты расширений
\usepackage[unicode,colorlinks=true,citecolor=black,linkcolor=black]{hyperref}
%\usepackage[pdftex,unicode,colorlinks=true,linkcolor=blue]{hyperref}
\usepackage{indentfirst} % включить отступ у первого абзаца
\usepackage[dvips]{graphicx} %хотим вставлять рисунки?
\graphicspath{{illustr/}}%путь к рисункам

\makeatletter
\renewcommand{\@biblabel}[1]{#1.} % Заменяем библиографию с квадратных скобок на точку:
\makeatother %Смысл этих трёх строчек мне непонятен, но поверим "Запискам дебианщика"

\usepackage{geometry} % Меняем поля страницы.
\geometry{left=2cm}% левое поле
\geometry{right=1cm}% правое поле
\geometry{top=2cm}% верхнее поле
\geometry{bottom=2cm}% нижнее поле

\renewcommand{\theenumi}{\arabic{enumi}}% Меняем везде перечисления на цифра.цифра
\renewcommand{\labelenumi}{\arabic{enumi}}% Меняем везде перечисления на цифра.цифра
\renewcommand{\theenumii}{.\arabic{enumii}}% Меняем везде перечисления на цифра.цифра
\renewcommand{\labelenumii}{\arabic{enumi}.\arabic{enumii}.}% Меняем везде перечисления на цифра.цифра
\renewcommand{\theenumiii}{.\arabic{enumiii}}% Меняем везде перечисления на цифра.цифра
\renewcommand{\labelenumiii}{\arabic{enumi}.\arabic{enumii}.\arabic{enumiii}.}% Меняем везде перечисления на цифра.цифра

\sloppy


\begin{document}

\setlength{\jot}{12pt}

Выполнил Николай Авдеев. Вариант 2.

\begin{figure}[ht]
	\includegraphics[width=\textwidth]{scheme.pdf}
	%\caption{}
\end{figure}

\paragraph{Обозначения.}
Левый конец ребра $\gamma_i$ будем обозначать через $\gamma_i^-$,
правый~--- через $\gamma_i^+$.
Значения функции в этих точках будем понимать в смысле соответствующего одностороннего предела.

\paragraph{Функционал равновесия.}
Выписываем сначала функционал энергии.
Энергия, накопленная в струнах, учитывается так же, как и в случае одной струны.
Энергия, накопленная в пружинах, учитывается по закону Гука.
\begin{multline}
	Vu = \sum_{i=1}^4 V_{i_{\mbox{\tiny струны}}}u + \sum_{i=1}^4 V_{i_{\mbox{\tiny пружины}}}u
	=
	\\=
	\int_{\Gamma} \frac{(p(x)u'(x))^2}{2} dx
	+ \frac{k_1}{2} (u(b_1))^2 + \frac{k_4}{2} (u(b_4))^2
	+ \frac{k_2}{2} (u(b_2)-u(a_4))^2 + \frac{k_3}{2} (u(b_2)-u(a_3))^2
\end{multline}
Функционал работы выписывается аналогично одномерному случаю:
\begin{equation}
	A_f u = \int_{\Gamma} f(x) u(x) dx
\end{equation}
Следовательно, функционал равновесия имеет вид:
\begin{multline}
	Ju =
	Vu - A_f u
	=
	\\=
	\int_{\Gamma} \frac{(p(x)u'(x))^2}{2} - f(x) u(x) dx
	+ \frac{k_1}{2} (u(b_1))^2 + \frac{k_4}{2} (u(b_4))^2
	+ \frac{k_2}{2} (u(b_2)-u(a_4))^2 + \frac{k_3}{2} (u(b_2)-u(a_3))^2
\end{multline}
Найдём теперь его первую вариацию.
В силу линейности первой вариации мы можем воспользоваться
выражением для первой вариации интегральной части, полученным ранее:

\end{document}
