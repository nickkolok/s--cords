\documentclass[a4paper,12pt]{article} %размер бумаги устанавливаем А4, шрифт 12пунктов
\usepackage[T2A]{fontenc}
\usepackage[utf8]{inputenc}
\usepackage[english,russian]{babel} %используем русский и английский языки с переносами
\usepackage{amssymb,amsfonts,amsmath,mathtext,enumerate,float,amsthm} %подключаем нужные пакеты расширений
\usepackage[unicode,colorlinks=true,citecolor=black,linkcolor=black]{hyperref}
%\usepackage[pdftex,unicode,colorlinks=true,linkcolor=blue]{hyperref}
\usepackage{indentfirst} % включить отступ у первого абзаца
\usepackage[dvips]{graphicx} %хотим вставлять рисунки?
\graphicspath{{illustr/}}%путь к рисункам

\makeatletter
\renewcommand{\@biblabel}[1]{#1.} % Заменяем библиографию с квадратных скобок на точку:
\makeatother %Смысл этих трёх строчек мне непонятен, но поверим "Запискам дебианщика"

\usepackage{geometry} % Меняем поля страницы.
\geometry{left=2cm}% левое поле
\geometry{right=1cm}% правое поле
\geometry{top=2cm}% верхнее поле
\geometry{bottom=2cm}% нижнее поле

\renewcommand{\theenumi}{\arabic{enumi}}% Меняем везде перечисления на цифра.цифра
\renewcommand{\labelenumi}{\arabic{enumi}}% Меняем везде перечисления на цифра.цифра
\renewcommand{\theenumii}{.\arabic{enumii}}% Меняем везде перечисления на цифра.цифра
\renewcommand{\labelenumii}{\arabic{enumi}.\arabic{enumii}.}% Меняем везде перечисления на цифра.цифра
\renewcommand{\theenumiii}{.\arabic{enumiii}}% Меняем везде перечисления на цифра.цифра
\renewcommand{\labelenumiii}{\arabic{enumi}.\arabic{enumii}.\arabic{enumiii}.}% Меняем везде перечисления на цифра.цифра

\sloppy


\begin{document}

\setlength{\jot}{12pt}

Выполнил Николай Авдеев. Вариант 2.

\begin{figure}[ht]
	\includegraphics[width=\textwidth]{scheme.pdf}
	%\caption{}
\end{figure}

\paragraph{Обозначения.}
Левый конец ребра $\gamma_i$ будем обозначать через $\gamma_i^-$,
правый~--- через $\gamma_i^+$.
Значения функции в этих точках будем понимать в смысле соответствующего одностороннего предела.

\paragraph{Функционал равновесия.}
Выписываем сначала функционал энергии.
Энергия, накопленная в струнах, учитывается так же, как и в случае одной струны.
Энергия, накопленная в пружинах, учитывается по закону Гука.
\begin{multline}
	Vu = \sum_{i=1}^4 V_{i_{\mbox{\tiny струны}}}u + \sum_{i=1}^4 V_{i_{\mbox{\tiny пружины}}}u
	=
	\\=
	\int_{\Gamma} \frac{p(x)(u'(x))^2}{2} dx
	+ \frac{k_1}{2} (u(a_1^1))^2
	+ \frac{k_2}{2} (u(b_2))^2
	+ \frac{k_3}{2} (u(a_2^3)-u(a_2^4))^2
	+ \frac{k_4}{2} (u(a_2^2)-u(a_2^4))^2
\end{multline}
Функционал работы выписывается аналогично одномерному случаю:
\begin{equation}
	A_f u = \int_{\Gamma} f(x) u(x) dx
\end{equation}
Следовательно, функционал равновесия имеет вид:
\begin{multline}
	Ju =
	Vu - A_f u
	=
	\\=
	\int_{\Gamma} \frac{p(x)(u'(x))^2}{2} - f(x) u(x) dx
	+ \frac{k_1}{2} (u(a_1^1))^2
	+ \frac{k_2}{2} (u(b_2))^2
	+ \frac{k_3}{2} (u(a_2^3)-u(a_2^4))^2
	+ \frac{k_4}{2} (u(a_2^2)-u(a_2^4))^2
\end{multline}
Найдём теперь его первую вариацию.
В силу линейности первой вариации мы можем воспользоваться
выражением для первой вариации интегральной части, полученным ранее:
\begin{multline}
	\delta\left( \int_{\Gamma} \frac{p(x)(u'(x))^2}{2} - f(x) u(x) dx, v \right)
	=
	\int_{\Gamma} p(x)u'(x)v'(x) - f(x) v(x) dx
	=
	\\=
	\left.\sum_{i=1}^{4} p(x) u'(x) v(x)\right|_{x=\gamma_i^-}^{x=\gamma_i^+}
	=
	\\ =
	  p(a_1^1) u'(a_1^1) v(a_1^1) - p(b_1  ) u'(b_1  ) v(b_1  )
	+ p(b_2  ) u'(b_2  ) v(b_2  ) - p(a_2^2) u'(a_2^2) v(a_2^2)
	+ \\
	+ p(b_3  ) u'(b_3  ) v(b_3  ) - p(a_2^3) u'(a_2^3) v(a_2^3)
	+ p(a_2^4) u'(a_2^4) v(a_2^4) - p(a_1^4) u'(a_1^4) v(a_1^4)
\end{multline}
Исходя из того, что
\begin{equation}
	\delta((u(a))^2, v) = 2 u(a) v(a)
\end{equation}
и
\begin{equation}
	\delta((u(a)-u(b))^2, v) = 2 (u(a) - u(b)) (v(a) - v(b))
\end{equation}
можно выписать, что
\begin{multline}
	\delta(Ju, v)
	=
	  p(a_1^1) u'(a_1^1) v(a_1^1) - p(b_1  ) u'(b_1  ) v(b_1  )
	+ p(b_2  ) u'(b_2  ) v(b_2  ) - p(a_2^2) u'(a_2^2) v(a_2^2)
	+ \\
	+ p(b_3  ) u'(b_3  ) v(b_3  ) - p(a_2^3) u'(a_2^3) v(a_2^3)
	+ p(a_2^4) u'(a_2^4) v(a_2^4) - p(a_1^4) u'(a_1^4) v(a_1^4)
	+ \\
	+ k_1  u(a_1^1) v(a_1^1)
	+ k_2  u(b_2  ) v(b_2  )
	+ k_3 (u(a_2^3) - u(a_2^4)) (v(a_2^3) - v(a_2^4))
	+ k_4 (u(a_2^2) - u(a_2^4)) (v(a_2^2) - v(a_2^4))
\end{multline}

Из условия жёсткого закрепления $\gamma_1$ в $b_1$ имеем
\begin{equation}
	u(b_1) = v(b_1) = 0.
\end{equation}

Из условия свободного конца $\gamma_3$ в $b_3$ имеем
\begin{equation}
	u'(b_3) = v'(b_3) = 0.
\end{equation}

Из условия сцепления струн $\gamma_1$ и $\gamma_4$ в точке $a_1$ имеем
\begin{equation}
	u(a_1^1) = u(a_1^4), ~~ v(a_1^1) = v(a_1^4)
\end{equation}

С учётом этих трёх условий имеем
\begin{multline}
	\delta(Ju, v)
	=
	  p(a_1^1) u'(a_1^1) v(a_1^1)
	+ p(b_2  ) u'(b_2  ) v(b_2  ) - p(a_2^2) u'(a_2^2) v(a_2^2)
	- \\
	                              - p(a_2^3) u'(a_2^3) v(a_2^3)
	+ p(a_2^4) u'(a_2^4) v(a_2^4) - p(a_1^4) u'(a_1^4) v(a_1^1)
	+ \\
	+ k_1  u(a_1^1) v(a_1^1)
	+ k_2  u(b_2  ) v(b_2  )
	+ k_3 (u(a_2^3) - u(a_2^4)) (v(a_2^3) - v(a_2^4))
	+ k_4 (u(a_2^2) - u(a_2^4)) (v(a_2^2) - v(a_2^4))
\end{multline}



Воспользуемся теперь произвольностью выбора $v$
и тем, что в стационарном состоянии $\delta(Ju, v) = 0$.
Обозначим $A = \{b_2, b_3, a_1^1, a_1^4, a_2^2, a_2^3, a_2^4\}$.
Возьмём сначала $v$ такую, что $v(A \setminus \{a_1^1\}) = 0$, $v(a_1^1) \neq 0$.
Тогда
\begin{multline*}
	0 = \delta(Ju, v) =
	  p(a_1^1) u'(a_1^1) v(a_1^1)
	- p(a_1^4) u'(a_1^1) v(a_1^1)
	+ k_1  u(a_1^1) v(a_1^1)
	= \\ =
	  p(a_1^1) u'(a_1^1) v(a_1^1)
	- p(a_1^4) u'(a_1^4) v(a_1^1)
	+ k_1  u(a_1^1) v(a_1^1)
	,
\end{multline*}
откуда
\begin{equation}
	  p(a_1^1) u'(a_1^1)
	- p(a_1^4) u'(a_1^4)
	+ k_1  u(a_1^1) = 0
\end{equation}

Возьмём $v$ такую, что $v(A \setminus \{a_2^2\}) = 0$, $v(a_2^2) \neq 0$.
Тогда
\begin{equation*}
	0 = \delta(Ju, v) =
	- p(a_2^2) u'(a_2^2) v(a_2^2)
	+ k_4 (u(a_2^2) - u(a_2^4)) v(a_2^2)
	,
\end{equation*}
откуда
\begin{equation}
	- p(a_2^2) u'(a_2^2) + k_4 (u(a_2^2) - u(a_2^4)) = 0
\end{equation}


Возьмём $v$ такую, что $v(A \setminus \{a_2^3\}) = 0$, $v(a_2^3) \neq 0$.
Тогда
\begin{equation*}
	0 = \delta(Ju, v) =
	                              - p(a_2^3) u'(a_2^3) v(a_2^3)
	+ k_3 (u(a_2^3) - u(a_2^4)) v(a_2^3)
	,
\end{equation*}
откуда
\begin{equation}
	- p(a_2^3) u'(a_2^3) + k_3 (u(a_2^3) - u(a_2^4)) = 0
\end{equation}



Возьмём $v$ такую, что $v(A \setminus \{a_2^4\}) = 0$, $v(a_2^4) \neq 0$.
Тогда
\begin{equation*}
	0 = \delta(Ju, v) =
	  p(a_2^4) u'(a_2^4) v(a_2^4)
	- k_3 (u(a_2^3) - u(a_2^4))  v(a_2^4)
	- k_4 (u(a_2^2) - u(a_2^4))  v(a_2^4)
	,
\end{equation*}
откуда
\begin{equation}
	p(a_2^4) u'(a_2^4) - k_3 (u(a_2^3) - u(a_2^4)) - k_4 (u(a_2^2) - u(a_2^4)) = 0
\end{equation}


Наконец, возьмём $v$ такую, что $v(A \setminus \{b_2\}) = 0$, $v(b_2) \neq 0$.
Тогда
\begin{equation*}
	0 = \delta(Ju, v) =
	  p(b_2  ) u'(b_2  ) v(b_2  )
	+ k_2  u(b_2  ) v(b_2  )
	,
\end{equation*}
откуда
\begin{equation}
	p(b_2  ) u'(b_2  ) + k_2  u(b_2  ) = 0
\end{equation}

\paragraph{Математическая модель.}
Таким образом, математической моделью исследуемой струнно-пружинной системы является
краевая задача, состоящая из уравнения
\begin{equation}
	-(p(x)u'(x))' = f(x)
\end{equation}
и следующих 8 условий:
\begin{gather}
	l_1 u = u(b_1) = 0
	\\
	l_2 u = p(b_2  ) u'(b_2  ) + k_2  u(b_2  ) = 0
	\\
	l_3 u = u'(b_3) = 0
	\\
	l_4 u = u(a_1^1) - u(a_1^4) = 0
	\\
	l_5 u = p(a_1^1) u'(a_1^1) - p(a_1^4) u'(a_1^4) + k_1  u(a_1^1) = 0
	\\
	l_6 u =
	\\
	l_7 u =
	\\
	l_8 u =
\end{gather}
\end{document}
